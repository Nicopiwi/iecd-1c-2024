\documentclass{article}
\begin{document}


\section{Teorico}
(b) El metodo de validacion cruzada tiene como funcion objetivo a:

$$ CV(h) = \frac{1}{n}\sum_{i=1}^n (Y_i - \widehat{m_h}^{-i}(X_i))^2 $$

para la minimizacion en $h$, donde el estimador $\widehat{m_h}^{-i}(.)$ se computa de la siguiente manera:

$$ \widehat{m_h}^{-i}(x) =  \sum_{j=1, j \neq i}^n Y_j \frac{g_j(x)}{\sum\limits_{t=1,t\neq i}^n g_t(x)}$$

Donde, para mayor legibilidad en las cuentas que vendran, escribimos $$ g_j(x) = K \left( \frac{X_j - x}{h}\right)$$

Notar que el estimador $\widehat{m_h}^{-i}(.)$ se computa como en la igualdad (1) del enunciado, pero sin la i-esima observacion, $(X_i, Y_i)$. En todo lo que sigue, tomaremos un $i \leq n$ fijo. Dado que este $i$ es arbitrario, lo siguiente que probaremos sera valido para la validacion cruzada sacando cualquiera de las observaciones.

$$$$
Queremos probar la igualdad (2):

$$\widehat{m_h}^{-i}(X_i) = \frac{\widehat{m_h}(X_i) - Y_i w_{i,h}(X_i)}{1 - w_{i,h}(X_i)}$$

Donde, en concordancia con la notacion del enunciado, llamamos:
$$w_{i,h}(x) = \frac{g_i(x)}{\sum\limits_{t=1}^n g_t(x)}$$

$$$$
Para probar (2), notemos que tenemos a $\widehat{m_h}^{-i}(x)$ escrito como una sumatoria sobre $j = 1, 2, .., n$ con $j\neq i$. Estamos sumando $Y_j$ mutiplicado por una fraccion, analizamos esa fraccion:

$$
\frac{g_j(x)}{\sum\limits_{t=1,t\neq i}^n g_t(x)} = \frac{g_j(x)}{ -g_i(x) +\sum\limits_{t=1}^n g_t(x)}
$$
\pagebreak

Multiplicamos por un $1$ convenientemente escrito, osea, $\sum\limits_{t=1}^n g_t(x)$ sobre si mismo:

$$\frac{g_j(x)}{ -g_i(x) +\sum\limits_{t=1}^n g_t(x)} =  \frac{g_j(x)}{\sum\limits_{t=1}^n g_t(x)} * \frac{\sum\limits_{t=1}^n g_t(x)}{ -g_i(x) +\sum\limits_{t=1}^n g_t(x)} = w_{j,h}(x) * \frac{\sum\limits_{t=1}^n g_t(x)}{ -g_i(x) +\sum\limits_{t=1}^n g_t(x)}$$

Notar que el segundo multiplicando no depende de $j$. Reescribamos a este multiplicando con una expresion mas amigable:

$$\frac{\sum\limits_{t=1}^n g_t(x)}{ -g_i(x) +\sum\limits_{t=1}^n g_t(x)} = \left( \frac{ -g_i(x) +\sum\limits_{t=1}^n g_t(x)}{\sum\limits_{t=1}^n g_t(x)}\right)^{-1} = \left( -\frac{g_i(x)}{\sum\limits_{t=1}^n g_t(x)} + \frac{\sum\limits_{t=1}^n g_t(x)}{\sum\limits_{t=1}^n g_t(x)} \right)^{-1} = \left( -w_{i,h}(x) + 1 \right)^{-1}$$

$$$$

Ahora si, con esta informacion, podemos reescribir a $\widehat{m_h}^{-i}(.)$ como:

$$\widehat{m_h}^{-i}(x) =  \sum_{j=1, j \neq i}^n Y_j \frac{g_j(x)}{\sum\limits_{t=1,t\neq i}^n g_t(x)} = \sum_{j=1, j \neq i}^n Y_j \frac{w_{j,h}(x)}{1-w_{i,h}(x)}$$

Sacamos convenientemente el divisor de la fraccion, que no depende de $j$, de la sumatoria:

$$\widehat{m_h}^{-i}(x) = \frac{1}{1-w_{i,h}(x)} \sum_{j=1, j \neq i}^n Y_j *w_{j,h}(x)$$

Sabiendo que $\widehat{m_h}(x) = \sum_{j=1}^n Y_j *w_{j,h}(x)$, podemos reescribir la sumatoria de arriba como $$\widehat{m_h}(x) - Y_i*w_{i,h}(x) = \sum_{j=1, j\neq i}^n Y_j *w_{j,h}(x)$$

Finalmente, juntando todo lo recien visto y evaluando $\widehat{m_h}^{-i}(x)$ en $X_i$, probamos la igualdad (2)

$$\widehat{m_h}^{-i}(X_i) = \frac{\widehat{m_h}(X_i) - Y_i*w_{i,h}(X_i)}{1-w_{i,h}(X_i)}$$

\pagebreak

Para probar la igualdad (3), que establece que
$$
CV(h) = \frac{1}{n}\sum_{i=1}^n \frac{(Y_i - \widehat{m_h}(X_i))^2 }{(1-w_{i,h}(X_i))^2}
$$
reemplazaremos lo obtenido en (2) en la definicion de $CV(h)$ que enunciamos al principio.

$$CV(h) = \frac{1}{n}\sum_{i=1}^n (Y_i - \widehat{m_h}^{-i}(X_i))^2 = 
$$
$$
= \frac{1}{n}\sum_{i=1}^n \left(Y_i - \frac{\widehat{m_h}(X_i) - Y_i*w_{i,h}(X_i)}{1-w_{i,h}(X_i)}\right)^2 
$$
$$
= \frac{1}{n}\sum_{i=1}^n \left(\frac{Y_i * (1 - w_{i,h}(X_i))}{1 - w_{i,h}(X_i)} - \frac{\widehat{m_h}(X_i) - Y_i*w_{i,h}(X_i)}{1-w_{i,h}(X_i)}\right)^2
$$
$$
= \frac{1}{n}\sum_{i=1}^n \left(\frac{Y_i - Y_i  *  w_{i,h}(X_i))}{1 - w_{i,h}(X_i)} - \frac{\widehat{m_h}(X_i) - Y_i*w_{i,h}(X_i)}{1-w_{i,h}(X_i)}\right)^2
$$
$$
= \frac{1}{n}\sum_{i=1}^n \left(\frac{Y_i - Y_i  *  w_{i,h}(X_i)) - \widehat{m_h}(X_i) + Y_i*w_{i,h}(X_i)}{1 - w_{i,h}(X_i)}\right)^2
$$
$$
= \frac{1}{n}\sum_{i=1}^n \left(\frac{Y_i - \widehat{m_h}(X_i)}{1 - w_{i,h}(X_i)}\right)^2
$$
Que es lo que queriamos probar.

\end{document}
